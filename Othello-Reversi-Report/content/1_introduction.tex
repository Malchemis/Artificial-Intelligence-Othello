\chapter{Introduction}
\label{chap:introduction}
Dans le cadre de ce \ac{TP}, nous avons abordé la conception et l'implémentation d'une \ac{IA} pour jouer au jeu d'Othello. Ce jeu de réflexion à deux joueurs sur un damier de 64 cases, avec des pions de deux couleurs, présente un défi intéressant pour l'IA en raison de la complexité de ses règles et de son espace de recherche combinatoire. L'objectif principal était de développer un algorithme capable de jouer contre un joueur humain ou contre une autre IA.

Nous explorerons dans un premier temps la modélisation du jeu et les structures de données utilisées pour représenter le plateau et les pions. Nous décrirons ensuite les algorithmes développés, en particulier l'algorithme Minimax, et plusieurs de ses variantes. Nous présenterons également une fonction d'évaluation pour mesurer la qualité des positions.

De plus, nous détaillerons la conception d'un classifieur pour évaluer la qualité d'une position. Nous expliquerons comment ce dernier a été entraîné et comment il se compare aux autres heuristiques.

Enfin, nous discuterons des résultats obtenus et des améliorations possibles.

