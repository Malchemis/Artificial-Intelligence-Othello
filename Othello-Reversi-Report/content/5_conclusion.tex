\chapter{Conclusion}
\label{chap:conclusion}

Ce projet a permis d'explorer en profondeur les mécanismes des algorithmes min-max et Alpha-Beta dans le contexte du jeu d'Othello, un environnement aux règles simples mais à la complexité combinatoire élevée. Nous avons développé et évalué plusieurs stratégies d'intelligence artificielle, notamment les approches positionnelle, absolue, mobilité et mixte, en les faisant jouer les unes contre les autres ainsi que contre des joueurs humains. Nos résultats montrent que l'algorithme Alpha-Beta, en particulier, offre des avantages significatifs en termes de réduction de la complexité temporelle et de l'espace de recherche, ce qui le rend préférable pour des profondeurs de recherche élevées.

L'analyse de la performance des différents agents a révélé que les stratégies mixtes, qui adaptent leur approche selon la phase du jeu, étaient particulièrement efficaces, combinant les avantages des approches positionnelles, de mobilité et absolues au fur et à mesure que la partie progresse. Cela souligne l'importance d'une adaptation stratégique flexible en réponse à l'évolution de l'état du jeu. Aussi, les stratégies positionnelles, prédominantes dans Mixte, ne peuvent pas être négligées, car elles sont elle-même très efficaces. La connaissance préalable du jeu est donc un atout majeur pour l'IA.

Les défis rencontrés, notamment liées à la gestion de la complexité exponentielle et l'ordre d'exploration des nœuds, nous ont conduit à introduire plusieurs optimisations, telles que la représentation en BitBoard, un algorithme efficace de calcul des coups valides, et l'implémentation de l'élagage Alpha-Beta. Ces techniques ont permis de réduire considérablement le temps de calcul nécessaire pour explorer l'arbre de jeu, sans compromettre la qualité des décisions prises par l'IA.

D'autres améliorations futures sont envisageables, comme l'exploration de techniques de tri préalable des coups ou l'intégration de l'apprentissage par renforcement pour améliorer la capacité d'adaptation et d'apprentissage de l'IA. Il serait par ailleurs très intéressant de finir l'implémentation du modèle classifieur et de le comparer aux heuristiques existantes, ainsi qu'à l'état de l'art tel qu'Edax.

En conclusion, ce projet a non seulement renforcé notre compréhension des algorithmes de jeu à deux joueurs et de leurs applications pratiques mais a également ouvert la voie à de futures recherches sur l'optimisation des stratégies d'IA dans des jeux de complexité similaire. Nous avons pu démontrer que des avancées significatives sont possibles, même dans des contextes où les ressources de calcul sont limitées, par une utilisation judicieuse des techniques d'élagage et de l'analyse heuristique.

Sur un plan plus personnel, j'ai particulièrement aimé ce projet, et vais continuer à mettre à jour le dépôt github associé. J'espère pouvoir continuer à faire plus de Théorie des Jeux et surtout plus d'\ac{IA}, peut-être sur d'autres jeux de plateau. J'ai appris beaucoup de choses, et j'espère que ce rapport vous a plu. Merci de m'avoir lu.