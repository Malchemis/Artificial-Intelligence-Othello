\chapter{Discussion}
\label{chap:discussion}

Dans ce chapitre, nous discutons des résultats obtenus dans le chapitre précédent. Nous commençons par expliciter les résultats de complexité temporelle et d'exploration de l'arbre de jeu. Ensuite, nous discutons des résultats de la performance des agents et l'analyse des parties.

\section{Complexité et exploration de l'arbre de jeu}
L'analyse comparative entre les algorithmes Negamax et Alpha-Beta montre des différences significatives dans leur performance à travers différentes stratégies et profondeurs de recherche.

\subsection{Analyse temporelle}
À la \textbf{profondeur 2}, l'algorithme Alpha-Beta surpasse déjà le Negamax traditionnel. Le moins bon résultat, la  stratégie Positionnelle avec élagage représente 86.94\% du temps de calcul de Negamax, indiquant une efficacité remarquable de l'Alpha-Beta dans la limitation de l'espace de recherche. Cette tendance est constante à travers les stratégies, comme le Mobilité et le Mixe, où les réductions de temps sont substantielles (54.55\% et 67.22\% du temps de Negamax respectivement). On remarque cependant que les Écart-types sont bien plus élevés ($\approx 1.5 \,\, \text{à} \,\, 2$ fois plus) pour l'Alpha-Beta que pour le Negamax. Cela suggère que l'Alpha-Beta est plus sensible à la configuration de l'arbre de jeu, et le gain à chaque itération peut varier considérablement. Il serait potentiellement possible de réduire cet écart s'il on parvenait à explorer les noeuds dans un ordre plus optimal, ce qui est une des améliorations possibles de l'algorithme Alpha-Beta.

En ce qui concerne la \textbf{profondeur 4}, les avantages de l'Alpha-Beta deviennent encore plus prononcés. Ici, les pourcentages de réduction du temps moyen de calcul sont impressionnants, allant jusqu'à un rapport de 13.03\% pour la stratégie Mobilité. Cela suggère que l'effet de l'élagage Alpha-Beta est particulièrement bénéfique à des profondeurs plus grandes, où la complexité exponentielle du Negamax le rendrait autrement impraticable. De plus, les écarts-types sont devenues plus stables. Nous remarquons également que la stratégie Absolue est très couteuse. Cela est problement dû à la nature symétrique du plateau de jeu, qui implique que plusieurs coups entrainent le même score, ce qui rend une discrimination difficile pour l'algorithme. Nous pouvons faire l'hypothèse que pour Absolue, le nombre d'élagage est très inférieur à celui des autres stratégies

Enfin, à la \textbf{profondeur 6}, les avantages de l'Alpha-Beta sur le Negamax traditionnel sont encore plus renforcés. Ceci est cohérent avec les observations précédentes : à mesure que la profondeur augmente, les optimisations apportées par l'élagage Alpha-Beta sont essentielles pour maintenir une performance acceptable. Une remarque notable est l'écart-type de la stratégie Absolue, qui est particulièrement très élevé. Une amélioration potentielle de l'algorithme serait de ne pas considérer les branches symétriques, ce qui réduirait le nombre de nœuds explorés et potentiellement le temps de calcul, ainsi que les variations de performance. Notre implématation actuelle prend en compte les noeuds déjà explorés, nous pourrions donc améliorer l'algorithme en ne considérant le noeud seulement si son symétrique n'a pas déjà été exploré.


\subsection{Exploration de l'arbre de jeu}
Nos statistiques d'exploration des nœuds suivent globalement la même tendance que les temps de calcul.

Les graphiques illustrant le nombre de nœuds explorés par coup montrent clairement que l'Alpha-Beta est bien plus performant que le Negamax. Comme illustré dans les figures \ref{fig:complexity_node_explored-2}, \ref{fig:complexity_node_explored-4}, et \ref{fig:complexity_node_explored-6}, les aires sous les courbes pour l'Alpha-Beta sont nettement plus petites, indiquant moins de nœuds explorés par rapport à Negamax, et ce, à tous les niveaux de profondeur. Plus encore, l'amélioration est plus significative à mesure que la profondeur augmente, ce qui est conforme à la théorie sous-jacente de l'Alpha-Beta. En effet, la réduction théorique du nombre de nœuds explorés est de l'ordre de $\sqrt{b^d}$, où $b$ est le facteur de branchement et $d$ la profondeur de recherche. Nos mesures, quant à elles trouvent une réduction similaire d'environ $\frac{1}{c}\sqrt{b^d}$, avec $c\leq8$ pour la profondeur 6 seulement. Pour les profondeurs inférieur, la réduction est bien plus faible, ce qui est attendu, puisqu'à ces niveaux, l'efficacité de l'élagage est moindre.

Les tableaux \ref{tab:node_explored_summary} et \ref{tab:node_explored_summary-2} fournissent une vue d'ensemble quantitative de cette analyse. Les valeurs exactes des réductions des nœuds explorés, permettent une évaluation précise et détaillée des performances algorithmiques pour chaque stratégie. Notamment, la stratégie Mobilité à la profondeur 6 révèle la plus grande réduction proportionnelle, soulignant potentiellement une synergie entre l'élagage et cette stratégie en particulier qui a pour objectif d'optimiser la différence du nombre de coups possibles entre Joueur actuel et adverse.

\subsection*{Conclusions}

Dans l'ensemble, ces résultats justifient grandement l'utilisation de l'élagage Alpha-Beta, en particulier pour des jeux avec un grand espace d'état comme c'est le cas pour de nombreux jeux de plateaux. L'efficacité accrue de l'Alpha-Beta se traduit par des temps de calcul plus rapides et moins de ressources nécessaires, ce qui est crucial dans des scénarios en temps réel ou des systèmes avec des contraintes de ressources.

C'est pourquoi, dans la partie suivante, nous nous concentrons sur les performances des agents, en utilisant l'Alpha-Beta comme algorithme de recherche.

\pagebreak
\section{Performance des agents et analyse des parties}

Dans cette section, nous discutons des résultats obtenus lors du championnat inter-configuration, en mettant l'accent sur les performances du joueur noir et blanc à diverses profondeurs. 

À noter, la stratégie Mixe utilise ici pendant les 30 premiers coups la stratégie Positionnelle, puis jusqu'à 50 la Mobilité, en finissant par la stratégie Absolue. Voyons pourquoi combiner les stratégies est optimal.

\subsection{Performance des Heuristiques}

De manière générale, quelque soit la couleur, la table d'heuristique 1 offre un meilleur gain\footnote{sauf pour la profondeur 4 du joueur Blanc qui est plus équilibré autour de 97.7\% et 90.55\% pour la stratégie Positionnelle et Mixe respectivement}. Nous remarquons par ailleurs que le gain augmente avec la profondeur (pour Positionnelle), commençant autour de 120\%, puis 124\% et finalement 147\% pour les profondeurs 2, 4 et 6 respectivement pour le joueur noir.

Cette corrélation avec les couleurs pourrait supposer que la table d'heuristique a été obtenue sur la base de parties jouées par un joueur noir, ou que les valeurs de la table d'heuristique sont plus adaptées pour le joueur noir. Cela pourrait être dû à la nature du jeu, où le joueur noir a l'avantage de jouer en premier dans le cas ou une bonne heuristique lui est fourni, ce qui lui permet de contrôler le plateau et de limiter les options de l'adversaire.

Également, nous remarquons que l'écart-type pour la stratégie Positionnelle est toujours bien plus faible que celle de Mixe. 

\subsection{Comparaison des Stratégies et Avantages par Couleur}

L'analyse des scores globaux pour chaque joueur révèle les tendances suivantes :

\begin{itemize}
    \item Le joueur noir affiche généralement de meilleurs scores avec la stratégie positionnelle à toutes les profondeurs, indiquant l'importance du contrôle du plateau.
    \item Le joueur blanc tire davantage parti des stratégies mixtes à des profondeurs plus élevées, ce qui peut être attribué à une combinaison efficace de plusieurs facteurs de jeu, comme la mobilité et le positionnement.
\end{itemize}

\subsection*{Conclusions}

En conclusion, les stratégies mixtes montrent un potentiel accru à des profondeurs d'analyse plus élevées, surtout pour le joueur blanc, tandis que les stratégies positionnelles sont avantageuses pour le joueur noir à des profondeurs moins élevées. Il est conseillé d'adapter la stratégie en fonction de la couleur et de la profondeur du jeu pour maximiser les performances.

\end{document}
